%	This subsection overviews the system identification methods employed for AtlantikSolar

Towards aiding the control synthesis procedure, a simplified linear state--space representation of the UAV dynamics was also derived based on recorded flight data and frequency--domain system identification methods. Around level flight, linear models may capture the vehicle response for small perturbations around a given equilibrium. Decoupling the longitudinal and lateral axis, the dynamics of a UAV may take the following form~\cite{dorobantu2011frequency,OMLAS_MED_14}: 

\small
\begin{eqnarray}\label{LON_DYN}
 \mathbf{M}_{lon}\dot{\mathbf{\hat{x}}}_{lon} &=& \mathbf{A}^\prime_{lon}\mathbf{\hat{x}}_{lon}+\mathbf{B}^\prime_{lon}\mathbf{u}_{lon} \\ \nonumber
 \mathbf{\hat{x}}_{lon} &=& \left[ \hat{u}~\hat{w}~\hat{q}~\hat{\theta} \right]^T
\end{eqnarray}
\normalsize
where $\hat{u},\hat{w},\hat{q},\theta$ correspond to the predicted body $x$--, $z$--axis velocities, the pitch rate and the pitch angle respectively, $\mathbf{u}_{lon} = [u_{elev}~u_{throt}]$ corresponds to the elevator deflection and the throttle command respectively, while

\footnotesize
\begin{eqnarray}
\mathbf{M}_{lon} &=& \begin{bmatrix}

m & 0 & 0 & 0\\ 
0 & m & 0 & 0\\ 
0 & 0 & I_y & 0\\ 
0 & 0 & 0 & 1
\end{bmatrix},~
\mathbf{B}^\prime_{lon} = \begin{bmatrix}
X_{u_{elev}} & X_{u_{throt}}\\ 
Z_{u_{elev}} & Z_{u_{throt}}\\ 
M_{u_{elev}} & M_{u_{throt}}\\ 
0
\end{bmatrix} \\ \nonumber
\mathbf{A}^\prime_{lon} &=& \begin{bmatrix}
X_u & X_w & X_q-mW_e & -mg\cos\theta_e\\ 
Z_u & Z_w & Z_q+mU_e & -mg\sin\theta_e\\ 
M_u & M_w & M_q & 0 \\ 
0 & 0 & 1 & 0
\end{bmatrix}
\end{eqnarray}
\normalsize
where $m$ is the mass, $I_y$ the inertia around the body $y$--axis, $W_e,\theta_e$ are the trimming points of vertical velocity and pitch, and the elements of $\mathbf{M}_{lon},\mathbf{A}^\prime_{lon}$ and $\mathbf{B}^\prime_{lon}$ form the stability and control derivatives of the longitudinal dynamics. 

Similarly, the lateral dynamics model takes the form:

\small
\begin{eqnarray}\label{LAT_DYN}
 \mathbf{M}_{lat}\dot{\mathbf{\hat{x}}}_{lat} &=& \mathbf{A}^\prime_{lat}\mathbf{\hat{x}}_{lat}+\mathbf{B}^\prime_{lat}\mathbf{u}_{lat} \\ \nonumber
 \mathbf{\hat{x}}_{lat} &=& \left[\hat{v}~\hat{p}~\hat{r}~\hat{\phi} \right]^T
\end{eqnarray}
\normalsize
where $\hat{v},\hat{p},\hat{r},\hat{\phi}$ correspond to the predicted body $y$--axis velocity, the roll and yaw rates and roll angle respectively, $\mathbf{u}_{lat}=[u_{ail}~u_{throt}]$ and $u_{ail}$ is the aileron deflection and $u_{rud}$ is the rudder deflection, while

\footnotesize
\begin{eqnarray}
\mathbf{M}_{lat} &=& \begin{bmatrix}
m & 0 & 0 & 0\\ 
0 & I_x & -I_{xz} & 0\\ 
0 & -I_{xz} & I_z & 0\\ 
0 & 0 & 0 & 1
\end{bmatrix},~ \mathbf{B}^\prime_{lat} = \begin{bmatrix}
Y_{u_{ail}} & Y_{u_{rud}}\\ 
L_{u_{ail}} & L_{u_{rud}}\\ 
N_{u_{ail}} & N_{u_{rud}}\\ 
0 & 0
\end{bmatrix}\\ \nonumber 
\mathbf{A}^\prime_{lat} &=& \begin{bmatrix}

Y_v & Y_p + mW_e & Y_r-mU_e & mg\cos\theta_e\\ 
L_v & L_p & L_r & 0\\ 
N_v & N_p & N_r & 0 \\ 
0 & 1 & \tan\theta_e & 0
\end{bmatrix}
\end{eqnarray}
\normalsize
where $I_x$ the inertia around the body $x$--axis, $I_{xz}$ the cross--inertia term of the $x,z$--axes and the elements of $\mathbf{M}_{lat},\mathbf{A}^\prime_{lat}$ and $\mathbf{B}^\prime_{lat}$ form the lateral stability and control derivatives. 

The frequency--domain system identification framework employed is based on the objective function and guidelines outlined in~\cite{TISCHLER_BOOK} and its specifics were further detailed in our previous work~\cite{OMLAS_MED_14}. Employing the same methods, models of sufficient fidelity were derived for the AtlantikSolar UAV. Figure~\ref{SysID_LonValid} depicts validation plots for the case of the longitudinal dynamics. As shown the model matching is  are derived while the increased coherence values indicate that the linearity assumption within such a conservative subset of the flight envelope is effectively correct. 

%
%%%%%%%%%%%%%%%%%%%%%%%%%%%%%%%%%%%%%%%%%%%%%%%%%%%%%%%%%%%%%%%%%%%%%%%%%%%%%%
\begin{figure}[htbp]
%\begin{center}
\centering
  \includegraphics*[width=8.5cm]{images/AtlantikSolar_SysID_Lon_Valid.eps}
%\end{center}
\caption{Longitudinal subsystem system identification validation plot. The angle of attack $\alpha$ is computed based on the experimental and model--predicted values for $u$ and $w$. Results for $1\textrm{h}$ are shown along with a closer look to $60\textrm{sec}$ of the flight and the coherence values $\gamma$ between the experimental $\alpha,q,\theta$ and the predicted $\hat{\alpha},\hat{q},\hat{\theta}$. }
\label{SysID_LonValid}
\end{figure}
%%%%%%%%%%%%%%%%%%%%%%%%%%%%%%%%%%%%%%%%%%%%%%%%%%%%%%%%%%%%%%%%%%%%%%%%%%%%%%
% 