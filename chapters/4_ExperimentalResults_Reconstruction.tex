%	This subsection presents some example results of the MeF ICARUS Field Testing for the AtlantikSolar System Overview paper
The AtlantikSolar is utilized within several research projects. Within that framework, it recently participated in the ICARUS project~\cite{ICARUSsite} field--trials event that took place in Marche--en--Famenne, Belgium. During these trials, several of the UAV capabilities were presented and especially its autonomous long--term operation as required by the SAR teams for collecting aerial data that can be utilized for mapping and reconnaissance purposes. The sensor pod described in Sec.~\ref{sec:detailed_design} was used on--board recording pose--annotated grayscale images. Furthermore, a GPS--tagged Sony HDR-AS100VW camera was also carried onboard. The UAV executed several pre--planned missions ensuring the complete coverage of a given map with its sensors and, with the help of the Pix4D software~\cite{Pix4Dsite}, the pose--annotated images enabled the dense reconstruction of $3\textrm{D}$ models of the area. Figure~\ref{fig:mef_icarus_reconstruction} presents such results, the reference waypoints and their radius of acceptance along with the recorded flight path and sectors of the dense point clouds. 


%
%%%%%%%%%%%%%%%%%%%%%%%%%%%%%%%%%%%%%%%%%%%%%%%%%%%%%%%%%%%%%%%%%%%%%%%%%%%%%%
\begin{figure}[htbp]
\begin{center}
  \includegraphics*[width=8.5cm]{images/MeF_GS_RGB_MergedResult_v8.eps} % 
\end{center}
\caption{Inspection mission reference waypoints and recorded path along with dense point clouds of sectors of the $3\textrm{D}$ reconstruction using the Aptina grayscale camera and the Sony RGB camera. The derived maps come with geoinformation while a small amount of highly--accurate ground control points (GCPs) can improve the georeferencing accuracy.  }
\label{fig:mef_icarus_reconstruction}
\end{figure}
%%%%%%%%%%%%%%%%%%%%%%%%%%%%%%%%%%%%%%%%%%%%%%%%%%%%%%%%%%%%%%%%%%%%%%%%%%%%%%
%
