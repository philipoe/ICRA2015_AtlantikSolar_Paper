%	This subsection presents some example results of the MeF ICARUS Field Testing for the AtlantikSolar System Overview paper
The AtlantikSolar is utilized within several research projects. Within that framework, AtlantikSolar recently participated in the ICARUS project~\cite{ICARUSwebsite} field--trials event that took place in Marche--en--Famenne, Belgium. During these trials, several of the platform capabilities were presented and especially its autonomous long--term operation as required by the SAR teams for collecting aerial data that can be utilized for mapping purposes and capturing thermal images for victim detection. The sensor pod mentioned in section~\ref{sec:detailed_design} was onboard AtlantikSolar recording pose--annotated visible--light grayscale and thermal images from an HDR global shutter camera (Aptina MT9V034) and a FLIR Tau2 respectively. Furthermore, a Sony HDR-AS100VW camera with GPS tagging was also carried onboard. The UAV executed several pre--planned missions ensuring the complete coverage of a given map with its sensors and, with the help of the Pix4D software~\cite{Pix4Dsite}, the pose--annotated images enabled the dense reconstruction of $3\textrm{D}$ models of the area. Figure~\ref{fig:mef_icarus_reconstruction} presents such results, the reference waypoints and their radius of acceptance along with the recorded flight path and segments of the densified point clouds. As shown, the vehicle tracks this relatively complex waypoint mission and the onboard sensory system allowed a very richfull perception of the area.

%
%%%%%%%%%%%%%%%%%%%%%%%%%%%%%%%%%%%%%%%%%%%%%%%%%%%%%%%%%%%%%%%%%%%%%%%%%%%%%%
\begin{figure}[htbp]
\begin{center}
  \includegraphics*[width=8.5cm]{images/MeF_GS_RGB_Merged_v1.eps} % 
\end{center}
\caption{Inspection mission reference waypoints and experimentally recorded path along with dense point clouds of sectors of the $3\textrm{D}$ reconstruction using the Aptina grayscale camera and the Sony RGB camera. The derived maps come with geoinformation while a small amount of highly--accurate ground control points (GCPs) can highly improve the georeferencing accuracy and also enable the seamless merging of different point clouds.  }
\label{fig:mef_icarus_reconstruction}
\end{figure}
%%%%%%%%%%%%%%%%%%%%%%%%%%%%%%%%%%%%%%%%%%%%%%%%%%%%%%%%%%%%%%%%%%%%%%%%%%%%%%
%

