%%%%%%%%%%%%%%%%%%%%%%%%%%%%%%%%%%%%%%%%%%%%%%%%%%%%%%%%%%%%%%%%%%%%%%%%%%%%%%%%
%2345678901234567890123456789012345678901234567890123456789012345678901234567890
%        1         2         3         4         5         6         7         8

\documentclass[letterpaper, 10 pt, conference]{ieeeconf}  % Comment this line out if you need a4paper

%\documentclass[a4paper, 10pt, conference]{ieeeconf}      % Use this line for a4 paper

\IEEEoverridecommandlockouts                              % This command is only needed if 
                                                          % you want to use the \thanks command

\overrideIEEEmargins                                      % Needed to meet printer requirements.

% See the \addtolength command later in the file to balance the column lengths
% on the last page of the document

% The following packages can be found on http:\\www.ctan.org
%\usepackage{graphics} % for pdf, bitmapped graphics files
%\usepackage{epsfig} % for postscript graphics files
%\usepackage{mathptmx} % assumes new font selection scheme installed
%\usepackage{times} % assumes new font selection scheme installed
%\usepackage{amsmath} % assumes amsmath package installed
%\usepackage{amssymb}  % assumes amsmath package installed

\title{\LARGE \bf
A solar-powered hand-launchable UAV for low-altitude multi-day continuous flight
}

\author{Philipp Oettershagen$^{1}$, Thomas Mantel, Amir Melzer, Stefan Leutenegger, Konstantinos Alexis$^{2}$ and Roland Siegwart% <-this % stops a space
\thanks{*This work was supported by a number of project partners and generous individuals, see http://www.atlantiksolar.ethz.ch/  }% <-this % stops a space
\thanks{All authors are part of the Autonomous Systems Lab, Swiss Federal Institute of Technology Zurich (ETH Zurich). Leonhardstrasse 21, 8092 Zurich, Switzerland. }
\thanks{$^{1}$
        {\tt\small Corresponding author, philipp.oettershagen@mavt.ethz.ch}}%
}

\begin{document}

\maketitle
\thispagestyle{empty}
\pagestyle{empty}

%%%%%%%%%%%%%%%%%%%%%%%%%%%%%%%%%%%%%%%%%%%%%%%%%%%%%%%%%%%%%%%%%%%%%%%%%%%%%%%%
\begin{abstract}

Abstract. Idea for this paper:

 - Conceptual design, realization/integration, development of onboard systems, flight testing and verification of conceptual/preliminary design => Complete cycle including all steps can be shown here.
   - Demonstrations -> rather basic control approaches chosen, i.e. this platform will be the basis for further research in control, guidance \& navigation, mapping and will go towards the applications of XXX
    - solar-powered, hand-launchable 5m-class Unmanned Aerial Vehicle with multi-day continuous flight capability combined with payload capacity for long-endurance SAR and inspection missions.
Questions:
      - This paper = engineering paper, rest is then BASING upon this paper (use it as a ref). Is this OK? Is the chance that this will be accepted big enough? -> Yes, focus on ``complete cycle'' here, with more details in papers XXX to YYY

\end{abstract}


%%%%%%%%%%%%%%%%%%%%%%%%%%%%%%%%%%%%%%%%%%%%%%%%%%%%%%%%%%%%%%%%%%%%%%%%%%%%%%%

 - We ware special : mission applications possible, long endurance, combination
\section{INTRODUCTION}



  (a) Background
    - Purpose\&applications\&advantages of solar-powered UAVs in general
  - Recent interest in HALE solar-powered UAVs -> Disadvantages? Or just lead to LALE UAVs as alternative
  - History/Preceeding research in solar UAVs (very brief)
      - recent papers in the same category. LALE solar UAVs. Then show that no similar solar-uav has been made?

  (b) this paper
   - Conceptual design, realization/integration, development of onboard systems, flight testing => Complete cycle including all steps can be shown here.
   - Based on  previous work at ETHZ/ASL (don't go into details)
   - LALE solar powered UAV, for use in Applications XXX,YYY,ZZZ, towards multi-day endurance, with advantages of XXX
   - Demonstrations -> rather basic control approaches chosen, i.e. this platform will be the basis for further research in control, guidance \& navigation, mapping and will go towards the applications of XXX
  
   Direct contributions of this paper:
1) conceptual design by us
2) platform realization (hardware, avionics/electronics) by us, 
3) estimation by us, rather basic pid control
3) tested/verified in extreme scenarios
   
   
   ------- LITERATURE ----

Other sUAVs (non-LALE)
 - Zephyr, Solar Impulse 1\&2, pathfinder, helios \& all NASA. Solara 50\&60 new, then the stuff that facebook bought.
Other LALE-sUAVs:
[1]  ``Solar Powered UAV: Design and Experiments'', S. Morton, IROS2014
[2] SoLong Design report, 2005, but manually flown
[3] SkySailor \& co.

Recheck: 
[a] A.J. Colozza: Preliminary Design of a Long Endurance Mars Aircraft, AIAA Paper 90-2000

\section{CONCEPTUAL DESIGN}
REFs: [1] Leutenegger@JIRS. [2] PhD Noth.
 - requirements and constraints.
 - market survey of basic technologies. Ref to Noth and so on? No, not here, but mention the underlying technologies briefly in the separate subchapters.
   - Aerodynamic considerations: Introduction to basic static level flight modelling of power consumption and stuff. Main REF to Noth and CO.
   - generalized results (a) t\_excess vs lat and mbat and stuff and (b) SoC margin w.r.t. cloud coverage factor and turbulence factor
   - Specific results (airframe results) for our requirements: wingspan, size, weight (as a result from the last section).
  
\section{DETAILED DESIGN AND REALIZATION}
(a) real hardware / Airframe design
  - Structure / how realized
  - CAD Model of
  	- whole plane
  	- structure in wings
  	- avionics installation \&implementation
   - Energy System: Bats\&Solar Power
   - weight distribution table (single parts, or better per component?)
(b) Propulsion
  - Propeller. Motor.
  - Test bench measurements?
(c) avionics
 + Overview Flowchart of components and interaction.
   - sensors and drivers
   - autopilot / Pixhawk
   - gps \& stuff
(d) payload
  - VI Sensor [ref to VI-sensor paper; ref to Leutenegger thesis?]
(e) comparison to conceptual design.

Onboard state estimation \& control
 - State estimation -> REF to stefan\&Amir paper
 - System Identification \& Modelling
 - Control using PID,  outer loops TECS \& L1 (Ref to OMLAS MED paper, also saying that there is future technologies which are being developed).
 - Full pre-flight verification in HIL
 
 (f) preliminary / low level results
  - control:   - SE\&Control: PID performance over various trim points. PID computational requirements (low!)
  - state estimation
 
 \section{EXPERIMENTAL RESULTS}
 
 [make this large, cause this is the main contribution?]

  - Power System
  - 12hrs battery powered flight -> power efficiency
  - 12hrs sensor flight -> with mppts for sure
  - 24hrs day/night flight
  - mapping missions in ICARUS. REF to Separate paper??? Yes, but only once both are accepted.
    
 Other (TBD)
  - Meteo planning? Nope. only mention as side note.
  - Regulations? Nope. only mention as side note.
   
\section{CONCLUSIONS}

A conclusion section is not required. Although a conclusion may review the main points of the paper, do not replicate the abstract as the conclusion. A conclusion might elaborate on the importance of the work or suggest applications and extensions. 

\addtolength{\textheight}{-12cm}   % This command serves to balance the column lengths
                                  % on the last page of the document manually. It shortens
                                  % the textheight of the last page by a suitable amount.
                                  % This command does not take effect until the next page
                                  % so it should come on the page before the last. Make
                                  % sure that you do not shorten the textheight too much.

%%%%%%%%%%%%%%%%%%%%%%%%%%%%%%%%%%%%%%%%%%%%%%%%%%%%%%%%%%%%%%%%%%%%%%%%%%%%%%%%



%%%%%%%%%%%%%%%%%%%%%%%%%%%%%%%%%%%%%%%%%%%%%%%%%%%%%%%%%%%%%%%%%%%%%%%%%%%%%%%%



%%%%%%%%%%%%%%%%%%%%%%%%%%%%%%%%%%%%%%%%%%%%%%%%%%%%%%%%%%%%%%%%%%%%%%%%%%%%%%%%
\section*{APPENDIX}

Appendixes should appear before the acknowledgment.

\section*{ACKNOWLEDGMENT}

The preferred spelling of the word �acknowledgment� in America is without an �e� after the �g�. Avoid the stilted expression, �One of us (R. B. G.) thanks . . .�  Instead, try �R. B. G. thanks�. Put sponsor acknowledgments in the unnumbered footnote on the first page.



%%%%%%%%%%%%%%%%%%%%%%%%%%%%%%%%%%%%%%%%%%%%%%%%%%%%%%%%%%%%%%%%%%%%%%%%%%%%%%%%

References are important to the reader; therefore, each citation must be complete and correct. If at all possible, references should be commonly available publications.



\begin{thebibliography}{99}

\bibitem{c1} G. O. Young, �Synthetic structure of industrial plastics (Book style with paper title and editor),� 	in Plastics, 2nd ed. vol. 3, J. Peters, Ed.  New York: McGraw-Hill, 1964, pp. 15�64.
\end{thebibliography}

\end{document}
